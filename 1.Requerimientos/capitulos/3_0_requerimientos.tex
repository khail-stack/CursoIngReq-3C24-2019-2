\chapter{Requerimientos}
\section{Requerimientos de interfaces externas}
Cuenta con dos perfiles uno para el alumno y otro para la empresa

\section{Requerimientos Funcionales}

Mediante los requerimientos funcionales se mostrára detalladamente la aplicacion realizada por modulos

\subsection{Inicio de Sesión}
\begin{flushright}
	\begin{enumerate}
		\setcounter{enumi}{100}
		\item El id del alumno sera el correo de Tecsup.
		\item El id de la empresa debe tener como minimo 8 caracteres.
		\item La clave del usuario debe ser un numero de 6 caracteres que debera ser cambiada cada 8 meses.
	\end{enumerate}
\end{flushright}
\subsection{Ofertas de Trabajo}
El area encargada de las ofertas laborales sera administrado por el area de pasantías
\begin{flushright}
	\begin{enumerate}
		\setcounter{enumi}{200}
		\item Las ofertas podrán ser visualizadas de acuerdo a su carrera.
		\item Podras enviar tus documentos a la empresa mediante nuestra plataforma.
		\item La empresa debera registrarse si desea publicar un anuncio de empleo.
		\item El alumno recibira anuncios de ofertas laborales a su plataforma.
	\end{enumerate}
\end{flushright}
\subsection{Crear Cuenta}
\begin{flushright}
	\begin{enumerate}
		\setcounter{enumi}{300}
		\item El usuario debera crear una cuenta en el cual debera especificar sus datos personales "Nombres,Apellidos, DNI, Edad, Carrera y el ciclo que cursa"
		\item Esta aplicacion solo será valida para los alumnos de III, IV, V ,VI y egresados de Tecsup .
	\end{enumerate}
\end{flushright}  

\section{Requerimientos No Funcionales}
\begin{flushright}
	\begin{enumerate}
		\setcounter{enumi}{300}
		\item No es una aplicacion internacinal.
		\item Ser responsables de los ofertas de trabajo.
	\end{enumerate}
\end{flushright}  



