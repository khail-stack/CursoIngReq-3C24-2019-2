\chapter{Requerimientos}
\section{Requerimientos de interfaces externas}
Cuenta con dos perfiles uno para el alumno y otro para la empresa

\section{Requerimientos Funcionales}

Mediante los requerimientos funcionales se mostrára detalladamente la aplicacion realizada por modulos

\subsection{Inicio de Sesión}
\begin{flushright}
	\begin{enumerate}
		\setcounter{enumi}{100}
		\item El alumno solo podrá iniciar sesión con el correo de Tecsup.
		\item La clave del usuario (empresa y/o alumno) será como mínimo de 6 caracteres.
		\item La empresa solo deberá registrarse si desea realizar una publicación.
		
	\end{enumerate}
\end{flushright}
\subsection{Ofertas de Trabajo}
El area encargada de las ofertas laborales sera administrado por el area de pasantías
\begin{flushright}
	\begin{enumerate}
		\setcounter{enumi}{200}
		\item La empresa deberá registrarse para publicar una oferta laboral.
		\item Las ofertas podrán ser visualizadas de acuerdo a la especialidad del alumno.
		\item El alumno podrá envíar su CV a través de la plataforma.
		\item El alumno recibirá anuncios de ofertas laborales a su plataforma.
	\end{enumerate}
\end{flushright}
\subsection{Crear Cuenta}
\begin{flushright}
	\begin{enumerate}
		\setcounter{enumi}{300}
		\item El usuario debera crear una cuenta en el cual debera especificar sus datos personales "Nombres,Apellidos, DNI, Edad, Carrera y el ciclo que cursa"
		\item Esta aplicacion solo será valida para los alumnos de III, IV, V ,VI y egresados de Tecsup .
	\end{enumerate}
\end{flushright}  

\section{Requerimientos No Funcionales}
\begin{flushright}
	\begin{enumerate}
		\setcounter{enumi}{300}
		\item No es una aplicacion internacinal.
		\item Ser responsables de los ofertas de trabajo.
	\end{enumerate}
\end{flushright}  



